\documentclass[]{article}
\usepackage{lmodern}
\usepackage{amssymb,amsmath}
\usepackage{ifxetex,ifluatex}
\usepackage{fixltx2e} % provides \textsubscript
\ifnum 0\ifxetex 1\fi\ifluatex 1\fi=0 % if pdftex
  \usepackage[T1]{fontenc}
  \usepackage[utf8]{inputenc}
\else % if luatex or xelatex
  \ifxetex
    \usepackage{mathspec}
  \else
    \usepackage{fontspec}
  \fi
  \defaultfontfeatures{Ligatures=TeX,Scale=MatchLowercase}
\fi
% use upquote if available, for straight quotes in verbatim environments
\IfFileExists{upquote.sty}{\usepackage{upquote}}{}
% use microtype if available
\IfFileExists{microtype.sty}{%
\usepackage{microtype}
\UseMicrotypeSet[protrusion]{basicmath} % disable protrusion for tt fonts
}{}
\usepackage[margin=1in]{geometry}
\usepackage{hyperref}
\PassOptionsToPackage{usenames,dvipsnames}{color} % color is loaded by hyperref
\hypersetup{unicode=true,
            pdftitle={PS1\_Small Business Loan Program},
            pdfauthor={Alexis Pearson},
            colorlinks=true,
            linkcolor=Maroon,
            citecolor=Blue,
            urlcolor=blue,
            breaklinks=true}
\urlstyle{same}  % don't use monospace font for urls
\usepackage{color}
\usepackage{fancyvrb}
\newcommand{\VerbBar}{|}
\newcommand{\VERB}{\Verb[commandchars=\\\{\}]}
\DefineVerbatimEnvironment{Highlighting}{Verbatim}{commandchars=\\\{\}}
% Add ',fontsize=\small' for more characters per line
\usepackage{framed}
\definecolor{shadecolor}{RGB}{248,248,248}
\newenvironment{Shaded}{\begin{snugshade}}{\end{snugshade}}
\newcommand{\AlertTok}[1]{\textcolor[rgb]{0.94,0.16,0.16}{#1}}
\newcommand{\AnnotationTok}[1]{\textcolor[rgb]{0.56,0.35,0.01}{\textbf{\textit{#1}}}}
\newcommand{\AttributeTok}[1]{\textcolor[rgb]{0.77,0.63,0.00}{#1}}
\newcommand{\BaseNTok}[1]{\textcolor[rgb]{0.00,0.00,0.81}{#1}}
\newcommand{\BuiltInTok}[1]{#1}
\newcommand{\CharTok}[1]{\textcolor[rgb]{0.31,0.60,0.02}{#1}}
\newcommand{\CommentTok}[1]{\textcolor[rgb]{0.56,0.35,0.01}{\textit{#1}}}
\newcommand{\CommentVarTok}[1]{\textcolor[rgb]{0.56,0.35,0.01}{\textbf{\textit{#1}}}}
\newcommand{\ConstantTok}[1]{\textcolor[rgb]{0.00,0.00,0.00}{#1}}
\newcommand{\ControlFlowTok}[1]{\textcolor[rgb]{0.13,0.29,0.53}{\textbf{#1}}}
\newcommand{\DataTypeTok}[1]{\textcolor[rgb]{0.13,0.29,0.53}{#1}}
\newcommand{\DecValTok}[1]{\textcolor[rgb]{0.00,0.00,0.81}{#1}}
\newcommand{\DocumentationTok}[1]{\textcolor[rgb]{0.56,0.35,0.01}{\textbf{\textit{#1}}}}
\newcommand{\ErrorTok}[1]{\textcolor[rgb]{0.64,0.00,0.00}{\textbf{#1}}}
\newcommand{\ExtensionTok}[1]{#1}
\newcommand{\FloatTok}[1]{\textcolor[rgb]{0.00,0.00,0.81}{#1}}
\newcommand{\FunctionTok}[1]{\textcolor[rgb]{0.00,0.00,0.00}{#1}}
\newcommand{\ImportTok}[1]{#1}
\newcommand{\InformationTok}[1]{\textcolor[rgb]{0.56,0.35,0.01}{\textbf{\textit{#1}}}}
\newcommand{\KeywordTok}[1]{\textcolor[rgb]{0.13,0.29,0.53}{\textbf{#1}}}
\newcommand{\NormalTok}[1]{#1}
\newcommand{\OperatorTok}[1]{\textcolor[rgb]{0.81,0.36,0.00}{\textbf{#1}}}
\newcommand{\OtherTok}[1]{\textcolor[rgb]{0.56,0.35,0.01}{#1}}
\newcommand{\PreprocessorTok}[1]{\textcolor[rgb]{0.56,0.35,0.01}{\textit{#1}}}
\newcommand{\RegionMarkerTok}[1]{#1}
\newcommand{\SpecialCharTok}[1]{\textcolor[rgb]{0.00,0.00,0.00}{#1}}
\newcommand{\SpecialStringTok}[1]{\textcolor[rgb]{0.31,0.60,0.02}{#1}}
\newcommand{\StringTok}[1]{\textcolor[rgb]{0.31,0.60,0.02}{#1}}
\newcommand{\VariableTok}[1]{\textcolor[rgb]{0.00,0.00,0.00}{#1}}
\newcommand{\VerbatimStringTok}[1]{\textcolor[rgb]{0.31,0.60,0.02}{#1}}
\newcommand{\WarningTok}[1]{\textcolor[rgb]{0.56,0.35,0.01}{\textbf{\textit{#1}}}}
\usepackage{graphicx,grffile}
\makeatletter
\def\maxwidth{\ifdim\Gin@nat@width>\linewidth\linewidth\else\Gin@nat@width\fi}
\def\maxheight{\ifdim\Gin@nat@height>\textheight\textheight\else\Gin@nat@height\fi}
\makeatother
% Scale images if necessary, so that they will not overflow the page
% margins by default, and it is still possible to overwrite the defaults
% using explicit options in \includegraphics[width, height, ...]{}
\setkeys{Gin}{width=\maxwidth,height=\maxheight,keepaspectratio}
\IfFileExists{parskip.sty}{%
\usepackage{parskip}
}{% else
\setlength{\parindent}{0pt}
\setlength{\parskip}{6pt plus 2pt minus 1pt}
}
\setlength{\emergencystretch}{3em}  % prevent overfull lines
\providecommand{\tightlist}{%
  \setlength{\itemsep}{0pt}\setlength{\parskip}{0pt}}
\setcounter{secnumdepth}{0}
% Redefines (sub)paragraphs to behave more like sections
\ifx\paragraph\undefined\else
\let\oldparagraph\paragraph
\renewcommand{\paragraph}[1]{\oldparagraph{#1}\mbox{}}
\fi
\ifx\subparagraph\undefined\else
\let\oldsubparagraph\subparagraph
\renewcommand{\subparagraph}[1]{\oldsubparagraph{#1}\mbox{}}
\fi

%%% Use protect on footnotes to avoid problems with footnotes in titles
\let\rmarkdownfootnote\footnote%
\def\footnote{\protect\rmarkdownfootnote}

%%% Change title format to be more compact
\usepackage{titling}

% Create subtitle command for use in maketitle
\providecommand{\subtitle}[1]{
  \posttitle{
    \begin{center}\large#1\end{center}
    }
}

\setlength{\droptitle}{-2em}

  \title{PS1\_Small Business Loan Program}
    \pretitle{\vspace{\droptitle}\centering\huge}
  \posttitle{\par}
    \author{Alexis Pearson}
    \preauthor{\centering\large\emph}
  \postauthor{\par}
      \predate{\centering\large\emph}
  \postdate{\par}
    \date{4/16/2020}

\usepackage{booktabs}
\usepackage{longtable}
\usepackage{array}
\usepackage{multirow}
\usepackage{wrapfig}
\usepackage{float}
\usepackage{colortbl}
\usepackage{pdflscape}
\usepackage{tabu}
\usepackage{threeparttable}
\usepackage{threeparttablex}
\usepackage[normalem]{ulem}
\usepackage{makecell}
\usepackage{xcolor}

\usepackage{float}
\usepackage{setspace}
\onehalfspacing

\begin{document}
\maketitle

\#\#PPHA 34600: Program Evaluation Spring 2020 \#\#Problem Set 1 \#\#Due
Thursday, April 23, at 9PM CT

My team and I have been asked by a well-meaning NGO, the Business
Underwriting and Loan International Group (BURLIG), to help them learn
about the impacts of their small business loan program on employee
income in California. BURLIG provides low-interest loans to small firms,
and hypothesizes that these loans are increasing incomes by raising
employment. The following report will work through the questions and
results our team helped to find and synthesize.

\#Question 1

BURLIG would like to know about the income impacts of their loans. They
say they're interested in measuring the impact of their loans, but don't
exactly know what that means. Here we have used the potential outcomes
framework to describe the impact of treatment (defined as ``taking a
small business loan''),\(Y_i(1)\) for firm \emph{i } on wages (measured
in dollars paid to workers) formally (meaning in math) and also in
words.

\textbf{Potential Outcome Framework}\\
*** This potential outcome framework allows us to understand and
visualize the causal effect of a binary treatment \(X_i\) on the outcome
\(Y_i\). For every individual \emph{i} there will exist two potential
outcomes depending on if the individual recieved treatment or not.\\
\(Y_i(1)\) is the outcome of individual \emph{i} if they recieve
treatment.\\
\(Y_i(0)\) is the outcome of individual \emph{i} if they do not recieve
treatment.\\
The observed outcomes of \(Y_i_\) can be written in terms of potential
outcomes \[\ Y_i = [Y_i(1) * X_i ]+ [Y_i(0)*(1-X_i)] \] or treatment
effects \[\ \tau_i = Y_i(1) - Y_i(0) \].

Due to the fact that two outcomes exist, this equation can then be
rewritten two ways depending on the treatment status of the individual.

\[\ Y_i = [Y_i(1) * X_i ]- [Y_i(0)*(1-X_i)] \]

\textbf{Treated}\\
\[\ Y_i = [Y_i(1) * (1) ]- [Y_i(0)*0] = Y_i(1)\]

In words: The potential impact of taking a small business loan, also
known as the treatment effect \[\tau_i\] , on firm \emph{i}'s wages is
equal to the potential impact on wages with the loan minus the potential
impact without the loan.

\textbf{Untreated}\\
\[\ Y_i = [Y_i(1) * 0 ]+ [Y_i(0)*1]= Y_i(0) \]

In words: The potential impact of not taking a small business loan on
firm \emph{i}'s wages is equal to the potential impact on wages with the
loan minus the potential impact without the loan.

\#Question 2 It is important to note that estimating \[\tau_i\] is
impossible. Like we explained earlier, an individual \emph{i} can exist
in two potential outcomes: the world where they recieve treatment and
the world where they don't. However, the \emph{Fundamental Problem of
Causal Inference} explains that we only get to observe one version of
individual \emph{i} and therefore we only get to see either the outcomes
of treatment, \(X(0)\) or no treatment \(X(1)\), we can't see both.
Because of this, there will always exist some counterfactual we can't
see at the same moment in time.

\#Question 3

Due to this issues of the missing counterfactual explained by the
\emph{Fundamental Problem of Causal Inference} we have to use other
methods to estimate the effect of treatment. A method that is sometimes
suggested is the \emph{Average Treatment Effect}. However, this will not
fully fix our issue. The \emph{Average Treatment Effect}, written\\
\[\ \tau^{ATE} = E[Y_i(1) - Y_i(0)], \]

is the difference in mean outcomes between units assigned to the
treatment and units assigned to the control. Simply an average of
\(\tau_i\) for \textbf{entire populations}/ \emph{all individuals}. It
is an expectations not using the real outcomes but the potential
outcomes. Because it is measuring singular individuals and because it is
not using actual outcomes, you are not able to definitively say that the
differences in outcomes we see is due to treatment and not simply some
other unobserved factor. We cannot observe the potential outcomes of a
singular individual.

If one was to the data provided by BURLIG, on firms with and without
loans, and wanted to compare the average wages paid across the two sets
of firms, in an effort to estimate the average treatment effect, they
would not end up with what they seek. In fact, instead of the average
treatment effect they would be calculated the \emph{Naive Estimator}.
The naive estimator is the difference between the AVERAGE TREATED
outcome on the \emph{treated} and the AVERAGE UNTREATED outcome on the
\emph{untreated}.

Visually that is written:
\[\ \tau^N = E[Y_i(1)|X_i = 1] - E[Y_i(0)|X_i = 0] \] where \(X_i\) is
the treatment variable.\\
Verbally:\\
The expected average wages when recieved a loan is equal to the expected
wages when recieved a loan conditional on recieving a loan minus
expected average wages when didn't recieve a loan conditional on not
recieveing the loan.

This is different from the \emph{Average Treatment Effect} because the
\emph{Naive Estimator} is assuming that everyone who recieves treatment
is the exact same as everyone who does not recieve treatment and it only
contains the observed outcomes.

Visually that is written:
\[\ E[Y_i(1)] = E[Y_i(1)|X_i = 1] = E[Y_i(1)|X_i = 0], \]\\
\[\ E[Y_i(0)] = E[Y_i(0)|X_i = 1] = E[Y_i(0)|X_i = 0]. \]

However, this is a bad assumption to make because most expectedly there
will be selection and depending if this selection is on the observables
or the unobservables, it may be hard to control for them and can skew
the story we are trying to tell.

\#Question 4

Another way to try and find the treatment effect is to find the
\emph{Average Treatment Effect on the Treated} or the \[\tau^{ATT}\].
The average treatment effect on the treated is written mathematically as
so: \[\ \tau^{ATT} = {E[Y_i(1)|D_i = 1]} - {E[Y_i(0)|D_i = 1]} \].
Essentially, we are taking the expected outcome of the treated
conditional on treatment and subtracting the expected outcome of
non-treatment conditional on treatment.\\
In the context of the BURLIG data we would be looking for the average
outcome on wages after recieving a loan if you recieved a loan
\[\ \underbrace{E[Y_i(1)|D_i = 1]}_\textrm{observed}\]and the average
outcome on wages after not recieving a loan if you recieved a loan-
\[\underbrace{E[Y_i(0)|D_i = 1]}_\textrm{potential/unobserved} \]. And
here inlies the issue, that second part is unobserved. If we are living
in the world of treatment it is impossible for us to see the effect of
non-treatment on someone who got treatment.

\#Question 5 BURLIG has conducted a pilot \emph{randomized } study to
estimate the effects of their loans on wages and we will start by
checking (with a proper statistical test) that the treatment group and
control group are balanced in pre-treatment wages (measured in dollars),
sales, number of workers, and owner gender. Note: Used \emph{burlig\_trt
} as your treatment variable. We check balance

\begin{Shaded}
\begin{Highlighting}[]
\NormalTok{bdata<-}\KeywordTok{read_csv}\NormalTok{(}\StringTok{"ps1_data.csv"}\NormalTok{)}
\end{Highlighting}
\end{Shaded}

\begin{verbatim}
## Parsed with column specification:
## cols(
##   burlig_trt = col_double(),
##   burlig_trt_take = col_double(),
##   baseline_employees = col_double(),
##   baseline_wages = col_double(),
##   baseline_sales = col_double(),
##   baseline_owner_female = col_double(),
##   endline_wages = col_double(),
##   endline_employees = col_double(),
##   endline_sales = col_double(),
##   endline_owner_female = col_double()
## )
\end{verbatim}

\begin{Shaded}
\begin{Highlighting}[]
\KeywordTok{View}\NormalTok{(bdata)}
\KeywordTok{summary}\NormalTok{(bdata)}
\end{Highlighting}
\end{Shaded}

\begin{verbatim}
##    burlig_trt     burlig_trt_take  baseline_employees baseline_wages      
##  Min.   :0.0000   Min.   :0.0000   Min.   : 8.496     Min.   :-1.750e+17  
##  1st Qu.:0.0000   1st Qu.:0.0000   1st Qu.:26.560     1st Qu.: 3.500e+04  
##  Median :1.0000   Median :0.0000   Median :29.994     Median : 3.738e+04  
##  Mean   :0.5002   Mean   :0.3284   Mean   :30.002     Mean   :-3.500e+13  
##  3rd Qu.:1.0000   3rd Qu.:1.0000   3rd Qu.:33.360     3rd Qu.: 1.319e+05  
##  Max.   :1.0000   Max.   :1.0000   Max.   :46.883     Max.   : 6.970e+05  
##  baseline_sales     baseline_owner_female endline_wages    
##  Min.   :18400000   Min.   :0.0000        Min.   :      0  
##  1st Qu.:19300000   1st Qu.:0.0000        1st Qu.: 945092  
##  Median :19500000   Median :0.0000        Median :1067187  
##  Mean   :19491500   Mean   :0.3652        Mean   :1075106  
##  3rd Qu.:19700000   3rd Qu.:1.0000        3rd Qu.:1192960  
##  Max.   :20500000   Max.   :1.0000        Max.   :2016082  
##  endline_employees endline_sales      endline_owner_female
##  Min.   : 8.496    Min.   :18500000   Min.   :0.0000      
##  1st Qu.:27.243    1st Qu.:19300000   1st Qu.:0.0000      
##  Median :30.684    Median :19500000   Median :0.0000      
##  Mean   :30.659    Mean   :19492800   Mean   :0.3652      
##  3rd Qu.:34.037    3rd Qu.:19700000   3rd Qu.:1.0000      
##  Max.   :48.883    Max.   :20600000   Max.   :1.0000
\end{verbatim}

\begin{Shaded}
\begin{Highlighting}[]
\CommentTok{#remove the outliers}
\NormalTok{bdata_}\DecValTok{2}\NormalTok{<-}\KeywordTok{subset}\NormalTok{(bdata,baseline_wages}\OperatorTok{>=}\DecValTok{0}\NormalTok{)}
\KeywordTok{summary}\NormalTok{(bdata_}\DecValTok{2}\NormalTok{)}
\end{Highlighting}
\end{Shaded}

\begin{verbatim}
##    burlig_trt     burlig_trt_take  baseline_employees baseline_wages  
##  Min.   :0.0000   Min.   :0.0000   Min.   : 8.496     Min.   : 15001  
##  1st Qu.:0.0000   1st Qu.:0.0000   1st Qu.:26.559     1st Qu.: 35000  
##  Median :1.0000   Median :0.0000   Median :29.993     Median : 37394  
##  Mean   :0.5001   Mean   :0.3283   Mean   :30.001     Mean   : 94276  
##  3rd Qu.:1.0000   3rd Qu.:1.0000   3rd Qu.:33.358     3rd Qu.:132010  
##  Max.   :1.0000   Max.   :1.0000   Max.   :46.883     Max.   :697037  
##  baseline_sales     baseline_owner_female endline_wages    
##  Min.   :18400000   Min.   :0.0000        Min.   : 296999  
##  1st Qu.:19300000   1st Qu.:0.0000        1st Qu.: 945108  
##  Median :19500000   Median :0.0000        Median :1067242  
##  Mean   :19491498   Mean   :0.3653        Mean   :1075321  
##  3rd Qu.:19700000   3rd Qu.:1.0000        3rd Qu.:1192986  
##  Max.   :20500000   Max.   :1.0000        Max.   :2016082  
##  endline_employees endline_sales      endline_owner_female
##  Min.   : 8.496    Min.   :18500000   Min.   :0.0000      
##  1st Qu.:27.242    1st Qu.:19300000   1st Qu.:0.0000      
##  Median :30.681    Median :19500000   Median :0.0000      
##  Mean   :30.658    Mean   :19492779   Mean   :0.3653      
##  3rd Qu.:34.036    3rd Qu.:19700000   3rd Qu.:1.0000      
##  Max.   :48.883    Max.   :20600000   Max.   :1.0000
\end{verbatim}

\begin{Shaded}
\begin{Highlighting}[]
\CommentTok{#Divide the data into treated and control }
\NormalTok{treated_bdata<-}\KeywordTok{subset}\NormalTok{(bdata_}\DecValTok{2}\NormalTok{,burlig_trt}\OperatorTok{==}\DecValTok{1}\NormalTok{)}
\NormalTok{control_bdata<-}\KeywordTok{subset}\NormalTok{(bdata_}\DecValTok{2}\NormalTok{, burlig_trt}\OperatorTok{==}\DecValTok{0}\NormalTok{)}

\CommentTok{#Balance Test}
\NormalTok{wagetest<-}\KeywordTok{t.test}\NormalTok{(}\DataTypeTok{x=}\NormalTok{treated_bdata}\OperatorTok{$}\NormalTok{baseline_wages,}\DataTypeTok{y=}\NormalTok{control_bdata}\OperatorTok{$}\NormalTok{baseline_wages)}
\NormalTok{salestest<-}\KeywordTok{t.test}\NormalTok{(}\DataTypeTok{x=}\NormalTok{treated_bdata}\OperatorTok{$}\NormalTok{baseline_sales,}\DataTypeTok{y=}\NormalTok{control_bdata}\OperatorTok{$}\NormalTok{baseline_sales)}
\NormalTok{employtest<-}\KeywordTok{t.test}\NormalTok{(}\DataTypeTok{x=}\NormalTok{treated_bdata}\OperatorTok{$}\NormalTok{baseline_employees,}\DataTypeTok{y=}\NormalTok{control_bdata}\OperatorTok{$}\NormalTok{baseline_employees)}
\NormalTok{femaletest<-}\KeywordTok{t.test}\NormalTok{(}\DataTypeTok{x=}\NormalTok{treated_bdata}\OperatorTok{$}\NormalTok{baseline_owner_female,}\DataTypeTok{y=}\NormalTok{control_bdata}\OperatorTok{$}\NormalTok{baseline_owner_female)}

\KeywordTok{stargazer}\NormalTok{(wagetest, }\DataTypeTok{title =} \StringTok{"Burlig Data Balance Tests"}\NormalTok{,}
          \DataTypeTok{dep.var.labels =} \StringTok{"Treated"}\NormalTok{, }\CommentTok{# renaming the dependent variable}
          \DataTypeTok{header =}\NormalTok{ F)                }\CommentTok{# get rid of the initial comments added by the author}
\end{Highlighting}
\end{Shaded}

\begin{verbatim}
## 
## % Error: Unrecognized object type.
\end{verbatim}

\begin{enumerate}
\def\labelenumi{\arabic{enumi}.}
\setcounter{enumi}{5}
\item
  Plot a histogram of wages for treated firms and control firms. What do
  you see? Re-do your balance table to reflect any necessary
  adjustments. What does this table tell you about whether or not
  BURLIG's randomization worked? What assumption do we need to make on
  unobserved characteristics in order to be able to estimate the causal
  effect of burlig\_trt  ?
\item
  Assuming that burlig\_trt  is indeed randomly assigned, describe how
  to use it to estimate the average treatment effect, and then do so.
  Please describe your estimate: what is the interpretation of your
  coefficient (be clear about your units)? Is your result statistically
  significant? Is the effect you find large or small, relative to the
  mean in the control group?
\item
  BURLIG is convinced that the reason their loans are effective is
  because they are leading firms to hire new workers. They want you to
  estimate the effects of their loans, but controlling for endline
  number of employees. Is this a good idea? Why or why not? Run this
  regression and describe your estimates. How do they differ from your
  results in (7)? What about controlling for baseline number of
  workers? Run this regression and describe your estimates. How do they
  differ from your results in (7)? How do the two estimates differ? What
  is driving any differences between them?
\item
  One of the BURLIG RAs (the real workforce!) informs you that not
  everybody who was assigned to treatment -- or was offered a loan --
  (burlig\_trt  = 1) actually took out  their loan. She tells you
  that the actual treatment indicator is burlig\_trt\_take  . (Since
  their loans were unique, we know for a fact that nobody in the control
  group got one). In light of this new information, what did you
  actually estimate in question (7)? How does this differ from what you
  thought you were estimating?
\item
  BURLIG aren't actually interested in the effect of assignment to
  treatment - they want to know about the actual effects of their loans.
  Describe (in math, and then in words) what you can estimate using the
  two treatment variables we observe, burlig\_trt  and
  burlig\_trt\_take  . Estimate this object (you can ignore standard
  errors just for this once). Interpret your findings. How does this
  compare to what you estimated in (7)?
\end{enumerate}


\end{document}
